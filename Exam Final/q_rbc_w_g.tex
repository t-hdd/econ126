\cleardoublepage
\section{Centralized RBC Model with Stochastic Government Consumption}

In this problem, you will simulate the dynamic equilibrium of a centralized RBC model without labor and with stochastic government consumption. As usual, the model features an infinitely-lived household that chooses consumption and capital accumulation to maximize the present value of its lifetime utility. What's new is that we'll assume that there is a government sector that consumes a stochastic quantity of goods each period. A key product of this modeling approach is that it allows us to model how fluctuations in government consumption affect the business cycle.



\subsection{The Model}

\subsubsection{Household Sector}

A representative household lives for an infinite number of periods. The expected present value of lifetime utility to the household from consuming $C_0, C_1, C_2, \ldots $ is denoted by $U_0$:
    \EE
    U_0 & = & E_0\sum_{t = 0}^{\infty} \beta^t \log (C_t),
    \FF	
where $0<\beta<1$ is the household's subjective discount factor. $E_0$ denotes the expectation with respect to all information available as of date 0.

The household enters period 0 with capital $K_0>0$. Production in period $t$ is according to a standard production function that has decreasing returns in capital $K_t$:
    \EE
    F(A_t,K_t) & = & A_t K_t^{\alpha}
    \FF
where TFP $A_t$ is stochastic:
    \EE
    \log A_{t+1} & = & \rho_A\log A_t + \epsilon^A_{t+1} \label{eqn:proj2_tfp}
    \FF
Each period the government collects a lump-sum tax $T_t$ from the household. The household's resource constraint in each period $t$ is therefore:
\EE
C_t + K_{t+1} + T_t & = &  A_t K_{t}^{\alpha}  + (1-\delta)K_t,
\FF
where $\delta$ is the rate of capital depreciation.

In period 0, the household solves:
    \EE
    && \max_{C_0,K_1} \; E_0\sum_{t=0}^{\infty}\beta^t\log (C_t) \nonumber\\
    && \; \; \; \;  \; \; \; \; \text{s.t.} \; \; \; \; C_t + K_{t+1} + T_t = A_tK_t^{\alpha} + (1-\delta)K_t
    \FF
which, as usual,  can be written as a choice of $K_1$ only:
    \EE
    \max_{K_1} \; E_0\sum_{t=0}^{\infty}\beta^t\log ( A_tK_t^{\alpha}  + (1-\delta)K_t - K_{t+1} - T_t)
    \FF

\subsubsection{Government Sector}


Each period the government consumes $G_t$ units of goods. $G_t$ evolves according to the following process:
	\EE
	\log G_{t+1} & = & (1-\rho_G)\log\bar{G}  + \rho_G \log G_{t} + \epsilon^G_{t+1} \label{eqn:proj2_gov}
	\FF
By assumption, the government always runs a balanced budget so:
	\EE
	T_t & = & G_t \label{eqn:proj2_gov_budget}
	\FF


\subsubsection{Investment and Output}

When the household chooses $K_{t+1}$, it implicitly chooses investment $I_t$ which is defined by:
	\EE
	I_t & = & K_{t+1} - (1-\delta)K_t \label{eqn:proj2_investment}
	\FF
and output $Y_t$ which is defined by:
    \EE
    Y_t & = & A_t K_t^{\alpha} \label{eqn:proj2_production}
    \FF

	
\subsubsection{Goods Market Clearing}

In equilibrium, the quantity of goods produced $Y_t$ has to equal the demand for those goods $C_t + I_t + G_t$:
	\EE
	Y_t & = & C_t + I_t + G_t\label{eqn:proj2_clearing}
	\FF
We call Equation (\ref{eqn:proj2_clearing}) the goods market clearing condition and it represents the aggregate resource constraint for the economy.

\subsubsection{Equilibrium}

The model has 7 endogenous variables: $A_t$, $G_t$, $K_t$, $C_t$, $T_t$, $Y_t$, $I_t$. Equilibrium is described by:
	\EN
	\item The household's first-order condition for $K_{t+1}$ (the Euler equation)
	\item The TFP evolution equation: Equation (\ref{eqn:proj2_tfp})
	\item The government consumption evolution equation: Equation (\ref{eqn:proj2_gov})
	\item The government budget constraint: Equation (\ref{eqn:proj2_gov_budget})
	\item The capital evolution equation: Equation (\ref{eqn:proj2_investment})
	\item The production function: Equation (\ref{eqn:proj2_production})
	\item The goods  market clearing equation: Equation (\ref{eqn:proj2_clearing})
	\NE

\subsubsection{Calibration}

Assume the following values for the model's parameters:
	
	\
	
	\begin{center}
	\begin{tabular}{ccl}\textbf{Parameter} & \textbf{Value} & \textbf{Description}\\\hline
	$\beta$		& 0.99		& household's subjective discount factor\\
	$\alpha$	& 0.35		& Cobb-Douglas production function parameter\\
	$\delta$	& 0.025 	& capital depreciation rate\\
	$\rho_A$	& 0.75		& autocorrelation of tfp\\
	$\sigma_A$	& 0.006		& s.d.~of TFP shock\\
	$\bar{G}$	& --		& steady state government consumption\\
	$\rho_G$	& 0.9		& autocorrelation of government consumption\\
	$\sigma_G$	& 0.015		& s.d.~of government consumption shock\\\hline
	\end{tabular}
	\end{center}

\subsection{Exercises}

\EN
\itemp{8} Download two series from FRED\footnote{\href{https://fred.stlouisfed.org/}{https://fred.stlouisfed.org/}}:
	\IZ
	\item[--] Government Consumption Expenditures and Gross Investment (Series ID: GCE)
	\item[--] Gross Domestic Product (Series ID: GCE)
	\ZI
Find the average of the ratio of government consumption to GDP for the US for all dates available.\footnote{Compute the ratio for each date \emph{first}, then compute the average of the ratio.} Report this value in your exam document.


\itemp{4} Solve for the household's first-order condition for $K_{t+1}$. Include this equation in your exam document and be able to explain the intuition behind it.

\itemp{6} Use Python to compute the steady state values of $A_t$, $K_t$, $Y_t$, and $I_t$. You will have to do this manually. Since you don't know $\bar{G}$ yet, you can't use \verb=linearsolve= for this step. Report the computed steady state values of $A_t$, $K_t$, $Y_t$, and $I_t$ in your exam document.

\itemp{6} Use the average ratio of government consumption to GDP for the US to \emph{calibrate} $\bar{G}$:
	\EE
	\bar{G} & = & \bar{Y} \times \Big[\text{Avg.~G-to-Y ratio}\Big]
	\FF
	Then use Python to compute the steady state values of $C_t$ and $T_t$. Report the computed steady state values of $G_t$, $C_t$, and $T_t$ in your exam document.

\itemp{6} Compute the impulse responses for all of the model's endogenous variables for 41 periods following a one percentage point increase in government consumption in period 5.\footnote{Note that like $A_t$, $G_t$ is a state (or predetermined) variable.} Create a set of clear, easy to read figures that depict the impulse responses and include them in your exam document. Units of plotted quantities should be percent deviations from steady state.\footnote{I.e., multiply the simulated impulse responses by 100.} Describe in words why the behavior of each variable in the simulated impulse responses.

\itemp{10} Make sure all code used to generate results for this problem is well-organized and thoroughly documented.
\NE